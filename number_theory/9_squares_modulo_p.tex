\section{Squares Modulo $p$}

So far, we only looked at $ax\equiv c \pmod{m}$. This chapter focuses on $x^{2}\equiv c \pmod{m}$.

\begin{definition}
(Quadratic Residue Modulo $p$ (QR) and NR) \\
A nonzero number that is congruent to a square modulo $p$ is called a quadratic residue modulo $p$. A number that is not congruent to a square modulo $p$ is called a (quadratic) nonresidue modulo $p$. QR and NR are abbreviations, respectively. 
\end{definition}

\noindent
For example, the full set of QRs modulo $7$ is $\{1, 2, 4\}$ and the full set of NRs modulo $7$ is $\{3, 5, 6\}$, as can be seen by the table.

\begin{table}[h!]
\centering
\begin{tabular}{||c c c c c c c c||} 
  \hline \hline
  x & 0 & 1 & 2 & 3 & 4 & 5 & 6 \\
  \hline
  c & 0 & 1 & 4 & 2 & 2 & 4 & 1 \\ 
  \hline \hline
\end{tabular}
\caption{Table of squares modulo 7}
%% \label{table:1}
\end{table}

\begin{theorem}
Let $p$ be an odd prime. There are exactly $\dfrac{p - 1}{2}$ QRs and NRs.
\end{theorem}

\begin{proof}
It is enough to show that $1^{2}, 2^{2},\dots (\dfrac{p - 1}{2})^{2} \pmod{p}$ are all different. \\
To prove, assume $\exists 1\leq b_{1}\neq b_{2}\leq \dfrac{p - 1}{2}$ such that $b_{1}^{2}\equiv b_{2}^{2} \pmod{p}$. \\
Then $p\mid b_{1}^{2} - b_{2}^{2} = (b_{1} + b_{2})(b_{1} - b_{2})$. \\
Note that $2\leq b_{1} + b_{2}\leq p - 1$, so $p\mid b_{1} - b_{2}$. This is a contradiction as there is no such $b_{1}\equiv b_{2} \pmod{p}$.

\begin{definition}
(Legendre Symbol of $a$ modulo $p$) \\
The Legendre symbol of $a$ modulo $p$ is $\legendre{a} = 1$ if $a$ is a quadratic residue modulo $p$, and $\legendre{a} = -1$ if $a$ is a quadratic nonresidue modulo $p$.
\end{definition}

\begin{theorem}
(Quadratic Residue Multiplication Rule) \\
Let $p$ be an odd prime. Then: \[\legendre{a} \legendre{b} = \legendre{ab}\]
\end{theorem}
\end{proof}

\noindent
The theorem implies that $QR\times QR = QR$, $QR\times NR = NR$, $NR\times NR = QR$.

\begin{exercise}
Compute $\legendre[97]{75}$
\end{exercise}

\begin{solution}
Our goal is to find $x$ such that $x^{2}\equiv 75 \pmod{97}$. \\
By Quadratic Residue Multiplication Rule, $\legendre[97]{75} = \legendre[97]{5^{2}}\legendre[97]{3} = \legendre[97]{3}$. \\
We observe that $10^{3}\equiv 3 \pmod{97}$, thus $\legendre[97]{3} = 1$. Hence, $\legendre[97]{75} = 1$.
\end{solution}