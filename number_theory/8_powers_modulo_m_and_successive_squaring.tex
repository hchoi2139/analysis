\section{$k$th Powers Modulo $m$ and $k$th Roots Modulo $m$}

Our goal is to compute $k$th powers modulo $m$ when $k$ and $m$ are very large. 

\begin{algorithm}
(Successive Squaring to Compute $a^{k} \pmod{m}$)
\begin{enumerate}
\item Write $k$ as a sum of powers of $2$, $k = u_{0} + u_{1}\cdot 2 + u_{2}\cdot 2^{2} + \cdots$, where each $u_{i}$ is either $0$ or $1$. (binary expansion of $k$.)
\item Make a table of powers of $a$ modulo $m$ using successive squaring. \\
$a^{1}\equiv A_{0}\pmod{m}$ \\
$a^{2}\equiv A_{0}^{2}\equiv A_{1}\pmod{m}$ \\
\vdots \\
$a^{2^{r}}\equiv A_{r - 1}^{2}\equiv A_{r}\pmod{m}$
\item The product $A_{0}^{u_{0}}\cdot A_{1}^{u_{1}}\cdots A_{r}^{u_{r}}\pmod{m}$ will be congruent to $a^{k}\pmod{m}$. 
\end{enumerate}
\end{algorithm}

\noindent
As an example, I compute $2^{9990}\pmod{9991}$. \\
First, I convert 9990 to binary: $9990 = 2^{13} + 2^{10} + 2^{9} + 2^{8} + 2^{2} + 2^{1}$. \\
Then, I look at the able of powers of $2$ modulo $m$:
\begin{table}[h!]
\centering
\begin{tabular}{||c c c c c c c||} 
  \hline \hline
  $2^{1}$ & $2^{2}$ & $2^{4}$ & $2^{8}$ & $2^{16}$ & $2^{32}$ & $2^{64}$ \\
  \hline
  2 & 4 & 16 & 256 & 5590 & 6243 & 158 \\ 
  \hline \hline
  $2^{128}$ & $2^{256}$ & $2^{512}$ & $2^{1024}$ & $2^{2048}$ & $2^{4096}$ & $2^{8192}$ \\
  \hline
  4982 & 2680 & 8862 & 5784 & 4788 & 5590 & 6243 \\
  \hline \hline
\end{tabular}
\caption{Table of Powers of $2$ modulo $m$}
%% \label{table:1}
\end{table}
\noindent
Finally, 
\begin{align*}
2^{9990} &\equiv 6243\cdot 5784\cdot 8862\cdot 2680\cdot 16\cdot 4 \pmod{9991} \\
&\equiv 2038\cdot 8862\cdot 2680\cdot 16\cdot 4 \pmod{9991} \\
&\equiv 7019\cdot 2680\cdot 16\cdot 4 \pmod{9991} \\
&\equiv 7858\cdot 64 \pmod{9991} \\
&\equiv 3362 \pmod{9991}
\end{align*}

\noindent
Now, we try to compute $k$th roots modulo $m$. In other words, suppose we are given $b$ and told to find $x$ such that $x^{k}\equiv b\pmod{m}$. We limit as $gcd(b, m) = 1$ and $gcd(k, \phi(m)) = 1$.

\begin{algorithm}
(Compute $k$th Roots Modulo $m$)
\begin{enumerate}
\item Compute $\phi(m)$. 
\item Find positive integers $u$ and $v$ such that $ku - \phi(m)v = 1$. In other words, find a positive integer $u$ such that $ku\equiv 1\pmod{\phi(m)}$.
\item Compute $b^{u}\pmod{m}$ by successive squaring.
\end{enumerate}
\end{algorithm}

\noindent
$gcd(k, \phi(m)) = 1$ is necessary as we can only find solutions $u$ and $v$ for $ku - \phi(m)v = 1$ under such limit. For better understanding, I provide an example: find $x$ which satisfies $x^{k}\equiv b\pmod{m}$. \\
First, compute $\phi(1073)$: $\phi(1073) = \phi(29)\phi(37) = 1008$. \\
Next, I find positive integer solutions to the equation $131u - 1008v = 1$ by Linear Equation Theorem. I first apply Euclidean algorithm and check that $gcd(131, 1008) = 1$: \\
\begin{align*}
1008 &= 131 \cdot 7 + 91 \\
131 &= 91 \cdot 1 + 40 \\
91 &= 40 \cdot 2 + 11 \\
40 &= 11 \cdot 3 + 7 \\
11 &= 7 \cdot 1 + 4 \\
7 &= 4 \cdot 1 + 3 \\
4 &= 3 \cdot 1 + 1
\end{align*}

\noindent
Now we back-trace and find $u$, $v$:
\begin{align*}
1 &= 4 - 3 = 4 - (7 - 4) \\
&= 4\cdot 2 - 7 = (11 - 7)\cdot 2 - 7 \\
&= 11\cdot 2 - 7\cdot 3 = 11\cdot 2 - (40 - 11\cdot 3)\cdot 3 \\
&= 11 \cdot 11 - 40 \cdot 3 = (91 - 40 \cdot 2) \cdot 11 - 40 \cdot 3 \\
&= 91 \cdot 11 - 40 \cdot 25 = 91 \cdot 11 - (131 - 91) \cdot 25 \\
&= 91 \cdot 36 - 131 \cdot 25 = (1008 - 131 \cdot 7) \cdot 36 - 131 \cdot 25 \\
&= 1008 \cdot 36 - 131 \cdot 277
\end{align*}

\noindent
By Linear Equation Theorem, the solution is expressible as $(u, v) = (-277 - 1008k, -36 - 131k)$. We thus can find positive integer solution $(731, 95)$. \\
Now I explain why we compute $758^{731} \pmod{1073}$, third step of \textbf{Algorithm 2.8.2}. \\
We observe that $(x^{131})^{731} = x^{131 \cdot 731} = x^{1 + 1008 \cdot 95} = x \cdot (x^{1008})^{95}.$ \\
As $\phi(1073) = 1008$, we know that $x^{1008}\equiv 1 \pmod{1073}$; hence, $x\equiv (x^{131})^{731}\equiv 758^{731} \pmod{1073}$. \\
Now we use successive squaring and compute $758^{731} \pmod{1073}$. Doing so, we acquire $x\equiv 905 \pmod{1073}$.