\section{Study on $-1$ and $2$ Square Modulo $p$}

We start by thinking which primes $p$ is $\legendre{-1} = 1$.

\begin{table}[h!]
\centering
\begin{tabular}{|c c c c c c c c c c c|} 
  \hline \hline
  p & 3 & 5 & 7 & 11 & 13 & 17 & 19 & 23 & 29 & 31 \\
  \hline
  Sol & NR & 2, 3 & NR & NR & 5, 8 & 4, 13 & NR & NR & 12, 17 & NR \\ 
  \hline \hline
\end{tabular}
\caption{Table of solutions to $x^{2}\equiv 1\pmod{p}$}
\end{table}

\noindent
It seems like if $p\equiv 1\pmod{4}$ then $-1$ is QR, and if $p\equiv 3\pmod{4}$ then $-1$ is NR. In fact, it is. To verify such tendency, we look for \textit{Square Root of Fermat's Little Theorem}. \\
\\
Let $A = a^{\frac{p - 1}{2}}\equiv \legendre{a}\pmod{p}$, $a\not\equiv 0\pmod{p}$. \\
Then by $A^{2}\equiv 1\pmod{p}$ by Fermat's Little Theorem. This means that $p\mid A^{2} - 1 = (A - 1)(A + 1)$. Hence, $p\mid A - 1$ or $p\mid A + 1$, which means $A\equiv 1\pmod{p}$ or $A\equiv -1\pmod{p}$.

\begin{theorem}
(Euler's Criterion) \\
Let $p$ be an odd prime. Then \[a^{(p - 1) / 2}\equiv \legendre{a}\pmod{p}.\]
\end{theorem}

\begin{proof}
Suppose that $a$ is QR, say $a\equiv b^{2}\pmod{p}$. Then, by Fermat's Little Theorem, \[a^{(p - 1) / 2}\equiv (b^{2})^{(p - 1) / 2} = b^{p - 1}\equiv 1\pmod{p}.\] 
Hence $a^{(p - 1) / 2}\equiv \legendre{a}\pmod{p}$. \\ 
We then should prove that all solutions to $x^{(p - 1) / 2}\equiv 1\pmod{p}$ are exactly same to the set of QRs. \\
We just proved that every QR is a solution to the congruence above, and there are exactly $\dfrac{p - 1}{2}$ distinct QRs. Also, by Polyomial Roots Mod $p$ Theorem, it has at most $\dfrac{p - 1}{2}$ distinct solutions. Hence, \[\{\text{solutions to } x^{(p - 1) / 2} - 1\equiv 0\pmod{p}\} = \{\text{quadratic residues modulo p}\}.\]
Now suppose that $a$ is NR. We know that $a^{p - 1}\equiv 1\pmod{p}$ by Fermat's Little Theorem. Thus \[0\equiv a^{p - 1} - 1\equiv (a^{(p - 1) / 2} - 1)(a^{(p - 1) / 2} + 1)\pmod{p}.\]
We know that $a^{(p - 1) / 2} - 1\not\equiv 0\pmod{p}$ as we previously showed the solutions to $x^{(p - 1) / 2} - 1\equiv 0\pmod{p}$ are the QRs. Hence, \[a^{(p - 1) / 2}\equiv -1 = \legendre{a}\pmod{p}.\]
\end{proof}

\noindent
Euler's Criterion clearly shows the tendency is true. As $\legendre{-1}\equiv (-1)^{(p - 1) / 2}$, if $p\equiv 1\pmod{4}$ then $(-1)^{(p - 1) / 2} = 1 = \legendre{-1}$; if $p\equiv 3\pmod{4}$ then $(-1)^{(p - 1) / 2} = -1 = \legendre{-1}$. \\
\\
\\
There is another interesting feature when inspecting the case of $2$ instead of $-1$. As we did previously, we look for primes $p$ such that $\legendre{2} = 1$.

\begin{table}[h!]
\centering
\begin{tabular}{|c||c c c c c c c c c c|} 
  \hline
  $p$ & 3 & 5 & 7 & 11 & 13 & 17 & 19 & 23 & 29 & 31 \\
  \hline
  $x^{2}\equiv 2$ & NR & NR & 3, 4 & NR & NR & 6, 11 & NR & 5, 18 & NR & 8, 23 \\ 
  \hline
  $p\pmod{8}$ & 3 & 5 & 7 & 3 & 5 & 1 & 3 & 7 & 5 & 7 \\
  \hline \hline
  $p$ & 37 & 41 & 43 & 47 & 53 & 59 & 61 & 67 & 71 & 73 \\
  \hline
  $x^{2}\equiv 2$ & NR & 17, 24 & NR & 7, 40 & NR & NR & NR & NR & 12, 59 & 32, 41 \\
  \hline
  $p\pmod{8}$ & 5 & 1 & 3 & 7 & 5 & 3 & 5 & 3 & 7 & 1 \\
  \hline
\end{tabular}
\caption{Table of solutions to $x^{2}\equiv 2\pmod{p}$ and $p\pmod{8}$}
\end{table}

\noindent
We observe the tendency that $\legendre{2} = 1 \text{ or } -1$, if $p\equiv 1, 7\pmod{8}$ or $p\equiv 3, 5\pmod{8}$, respectively. Unfortunately, we cannot use Euler's Criterion as calculating $2^{(p - 1) / 2}\pmod{p}$ is not easy. However, Gauss came up with a brilliant idea: \\
Say that we want to check if $2$ is a quadratic residue modulo $17$. Then we find that 

\begin{align*}
2\cdot 4\cdot 6\cdot 8\cdot 10\cdot 12\cdot 14\cdot 16 &= (2\cdot 1)(2\cdot 2)(2\cdot 3)(2\cdot 4)(2\cdot 5)(2\cdot 6)(2\cdot 7)(2\cdot 8) \\
&= 2^{8} \cdot 8!.
\end{align*}

\begin{align*}
2\cdot 4\cdot 6\cdot 8\cdot 10\cdot 12\cdot 14\cdot 16 &\equiv 2\cdot 4\cdot 6\cdot 8\cdot (-7)\cdot (-5)\cdot (-3)\cdot (-1) \\
&\equiv (-1)^{8}\cdot 8! \pmod{17}
\end{align*}

so $2^{8}\equiv 1\pmod{17}$, thus $2$ is a quadratic residue modulo $17$. \\
Generalising this idea, we can reach \[2^{(p - 1) / 2}\equiv (-1)^{(\text{number of integers in }\{2, 4, 6, \dots , p - 1\} \text{larger than (p - 1) / 2.})} \pmod{p}\]
Using this idea, it is not difficult to verify the tendency of $\legendre{2}$.