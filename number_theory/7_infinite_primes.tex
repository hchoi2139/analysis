\section{Infinite Primes}

I start by introducing \textbf{Prime Divisibility Theorem} and the \textbf{Fundamental Theorem of Arithmetic}.

\begin{theorem}[Prime Divisibility Theorem]
Suppose that for some prime $p$, if $p\mid a_{1}a_{2}\cdots a_{r}$ where $a_{1}a_{2}\cdots a_{r}\in \mathbb{Z}_{\geq 2}$, $r\in \mathbb{Z}_{\geq 2}$, then $1\leq \exists i\leq r$, $p\mid a_{i}$.
\end{theorem}

\noindent
I omit the proof.

\begin{theorem}[The Fundamental Theorem of Arithmetic]
All natural numbers larger than 1 can be factored into the product of primes in exactly one way.
\end{theorem}

\begin{proof}
I first prove the existence of the factorisation. I prove by induction. \\
We check the base case: the theorem holds in case of $2$. \\
We check the inductive case: assume that for $2\leq \forall n\leq k$, $\forall k\in \mathbb{Z}_{\geq 2}$ the theorem holds. \\
If $n = k + 1$ is prime, the theorem holds. \\
If $n = k + 1$ is composite, then we can factorise as $n = n_{1}n_{2}$ where $2\leq n_{1}, n_{2}\leq k$. \\
By assumption, $n_{1}, n_{2}$ can be facotored into the product of primes, so $n$ can. \\
\\
Now I prove the uniqueness of factorisation. \\
Suppose there are more than one way of factorisation: $n = p_{1}p_{2}\cdots p_{r} = q_{1}q_{2}\cdots q_{s}$. \\
Without loss of generality, say that $p_{1}p_{2}\cdots p_{r}$ and $q_{1}q_{2}\cdots q_{s}$ does not have common factor. \\
If they have some, simply erase them and acquire the same form. \\
Now look $p_{1}$. Since $p_{1}\mid p_{1}p_{2}\cdots p_{r} = n$, $p_{1}\mid q_{1}q_{2}\cdots q_{s}$. \\
Then, by \textbf{Prime Divisibility Theorem}, $1\leq \exists i\leq s$, $p_{1}\mid q_{i}$. Contradiction.
\end{proof}

\begin{theorem}
There are infinitely many primes.
\end{theorem}

\begin{proof}
  Suppose there are finite number of primes. \\
  Name each as $p_{1}, p_{2},\dots p_{n}$ where $n$ is the number of primes. \\
  Now we inspect the number $x = p_{1}p_{2}\cdots p_{n} + 1$. \\
  We observe that $x$ is a composite number as it is not one of the primes we pre-defined. \\
  Since $x$ is a composite number, according to \textbf{Fundamental Theorem of Arithmetic}, there must exist some prime number $p$ such that $p\mid x$. \\
  However, we observe there is no such prime number in our predefined prime numbers that can factorise $x$. This is a contradiction as, according to \textbf{Fundamental Theorem of Arithmetic}, all number should be factorised into the product of prime numbers. $x$ is a composite number, so there must exist some prime that can factorise $x$.
\end{proof}

\noindent
I provide an example regarding \textbf{Theorem 2.7.3} and generalise later to \textbf{Dirichlet's Theorem on Primes in Arithmetic Progressions}. 

\begin{exercise}
Show that there are infinitely many primes that are congruent to $5 \pmod{6}$.
\end{exercise}

\begin{solution}
Suppose that $ps = \{5, p_{1}, p_{2}\dots p_{n}\}$ be the finite primes which are congruent to 5 mod 6. \\
We inspect the number $x = 6p_{1}p_{2}\cdots p_{n} + 5$. \\ 
First observe that $x\equiv 5 \pmod{6}$. Since $x$ is not in pre-defined set $ps$, $x$ is composite. \\
Then, by \textbf{Fundamental Theorem of Arithmetic}, we can factorise $x$ into product of primes: $x = q_{1}q_{2}\cdots q_{r}$ where all $q$s are primes. \\
Now, observe that $3\nmid x$. Thus, $x = 6k + 1$ or $x = 6k + 5$. \\
Now we look at the features of $q$s. \\
If $\forall i$ $q_{i}\equiv 1 \pmod{6}$, then $x = q_{1}\cdots q_{r}\equiv 1 \pmod{6}$. We can thus know that at least one $q$ is $\equiv 5 \pmod{6}$. Say such $q$ as $q'$. \\
Then $q'\in ps$ as $ps$ contain all the primes that is $\equiv 5 \pmod{6}$. \\
If $q' = 5$, then $5\mid x$. Contradiction. \\
If $q' = p_{t}$, $1\leq t\leq n$, then $q'\mid 6p_{1}p_{2}\cdots p_{r}$ and $q'\mid x$. \\
So $q'\mid 5$, which is a contradiction.
\end{solution}

\begin{theorem}[Dirichlet's Theorem on Primes in Arithmetic Progressions]
Let $a, m\in \mathbb{Z}$, $gcd(a, b) = 1$. Then there are infinitely many primes that are congruent to $a \pmod{m}$.
\end{theorem}

\noindent
I omit the proof. \\
\\
Finally, I mention \textbf{Prime Number Theorem} and move on to next section.

\begin{theorem}[The Prime Number Theorem]
When $x\in \mathbb{Z}$ is large, the number of primes less than $x$ is approximately equal to $\dfrac{x}{ln{x}}$
\end{theorem}