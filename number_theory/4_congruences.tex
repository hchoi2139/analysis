\section{Congruences}

\begin{definition}
  $a$ is congruent to $b$ modulo m if $m\mid a - b$ and denote $a\equiv b\; (\bmod{m})$
\end{definition}

\noindent
It is noteworthy to mention some properties:

\begin{observation}
  If $a_{1}\equiv b_{1}\; (\bmod{m})$ and $a_{2}\equiv b_{2}\; (\bmod{m})$, then $a_{1} \pm a_{2}\equiv b_{1}\pm b_{2}\; (\bmod{m})$ and $a_{1}a_{2}\equiv b_{1}b_{2}\; (\bmod{m})$.
\end{observation}

\begin{observation}
  If $ac\equiv bc\; (\bmod{m})$, it need not be true that $a\equiv b\; (\bmod{m})$. If, however, $gcd(c, m) = 1$, it is always true. 
\end{observation}

\noindent
Now I should introduce a technique which we will (hopefully) love. \\
A \textbf{Climb Every Mountain Technique} is, when solving a congruence modulo $m$, we try each value $0, 1, \dots m - 1$. I would say this technique, in Korean, as No-ga-da. This technique is often useful and is only the technique you can use. For example, to solve $x$ such that $x^{2}\equiv 3\; (\bmod{10})$, we substitute $0\dots 9$ to $x$ and find out there is no solution.

\begin{theorem}[Linear Congruence Theorem]
  Let $a, c, m\in \mathbb{Z}$ with $m\geq 1$ and let $g = gcd(a, m)$. \\
  (1) If $g\nmid c$, then $\nexists{x}$ s.t. $ax\equiv c\; (\bmod{m})$. \\
  (2) If $g\mid c$, then $ax\equiv c\; (\bmod{m})$ has exactly $g$ in congruent solutions.
\end{theorem}

\begin{proof}
Proof for (1). \\
Suppose $\exists{x_{0}}$ such that $ax_{0}\equiv c\; (\bmod{m})$ when $g\nmid c$. \\
Then $\exists{y}$ such that $ax_{0} + my = c$. \\
Now observe that $g\mid a$ thus $g\mid ax_{0}$ and $g\mid m$ thus $g\mid my$. \\
Then $g\mid c$ should satisfy, and this is a contradiction. \\
I omit the proof for (2).
\end{proof}

Arguably, the most important case of \textbf{Linear Congruence Theorem} is when $gcd(a, m) = 1$: $ax\equiv c\; (\bmod{m})$. In this case, it has only one solution and denote it as $x\equiv a^{-1}c\; (\bmod{m})$. \\
\\
I lastly mention \textbf{Polynomial Roots Mod} $p$ \textbf{Theorem} and finish this section.

\begin{theorem}
(Polynomial Roots Mod $p$ Theorem) \\
Let $p$ be a prime number and let $f(x) = a_{0}x^{d} + a_{1}x^{d - 1} + \cdots + a_{d}$ be a polynomial of degree $d\geq 1$ with integer coefficients and with $p\nmid a_{0}$. Then $f(x)\equiv 0\pmod{p}$ has at most $d$ incongruent solutions.
\end{theorem}

\noindent
I omit the proof.